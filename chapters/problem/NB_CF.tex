\subsection{User-Based Top-N recommendation}
Phương pháp này dựa trên phương pháp user-based top-N recommendation ở trên nhưng đã được điều chỉnh để phù hợp cho bộ dữ liệu Movielens. Các bước thực hiện của phương pháp này như sau. Đầu tiên, tìm k user giống nhất cho active user sử dụng phương pháp cosine similarity. Sau đó, tìm set C bao gồm tất cả những item mà k user này đã đánh giá cùng với các rating của mỗi user cho mỗi item đó. Sau đó loại bỏ các item trùng nhau trong set C, những item còn lại sẽ gồm các giá trị trung bình rating của của những user trong k user cho những item này. Sau đó sắp xếp set C theo giá trị giảm dần của trung bình các rating. Lấy N giá trị đầu tiên ta có top-N recommendation.

\subsection{Item-Based Top-N recommendation}
Phương pháp này dựa trên phương pháp item-based top-N recommendation ở trên, đã được điều chỉnh để phù hợp cho bộ dữ liệu Movielens. Các bước thực hiện phương pháp này như sau. Đầu tiên, tìm k item giống nhất cho mỗi item dựa trên cách tính cosine similarity. Sau đó tìm set C, là hợp của các tập k item giống nhất với những item mà active user thích (có rating > 0 sau khi đã chuẩn hóa). Sau đó xóa những item mà active user và item bị trùng lặp, những item còn lại bao gồm các số lần xuất hiện trong set C. Sắp xếp set C theo giá trị xuất hiện giảm dần, lấy N item đầu tiên, ta được top-N recommendation.

\subsection{Top-N recommendation dựa trên rating dự đoán}
Phương pháp này là phương pháp tìm top-N recommendation cho active user dựa trên việc tính toán rating của active user đó cho tất cả các item. Bước đầu tiên của phương pháp này là áp dụng phương pháp user-based collaborative filtering để dự đoán rating của user đó cho mỗi item. Bước tiếp theo, tìm ra N item mới được dự đoán rating có giá trị rating lớn nhất, ta có được top-N recommendation. Tuy nhiên phương pháp này gặp vấn đề performance vì để tìm được top-N recommendation cho active user, cần phải dự đoán tất cả rating của user đó cho tất cả các item.